%---------------------------------------------------------------------------------------------------
% Einstellungen
% (gelten nur in Zusammenarbeit mit pdflatex)
%---------------------------------------------------------------------------------------------------
\documentclass[
  pagesize,	                                           % flexible Auswahl des Papierformats
  a4paper,  	                                         % DIN A4
  oneside,    	                                       % einseitiger Druck
  BCOR5mm,      	                                     % Bindungskorrektur
  headsepline,                                         % Strich unter der Kopfzeile
  12pt,                                                % 12pt Schriftgr��e
	halfparskip,                                         % Europ�ischer Satz: Abstand zwischen Abs�tzen
	abstracton,																					 % Spezielle Formatierung, die erlaubt, dass die
																											 % Zusammenfassung vor dem Inhaltsverzeichnis steht
	%draft,																							 % Es handelt sich um eine Vorabversion	
	final,																							 % Es handelt sich um die endg�ltige Version
	liststotoc,																					 % Tabellen- und Abbildungsverzeichnis im 																																 % Inhaltsverzeichnis
	idxtotoc,																						 % Index im Inhaltsverzeichnis	
  bibtotoc,                                            % Literaturverzeichnis im Inhaltsverzeichnis  
]{scrreprt}                                            % KOMA-Scriptklasse Report

%---------------------------------------------------------------------------------------------------
\usepackage[english,ngerman]{babel}                    % deutsche Trennmuster
\usepackage[T1]{fontenc}                               % EC-Schriften, Trennstellen nach Umlauten
\usepackage[latin1]{inputenc}                          % direkte Umlauteingabe (� statt "a)
                                                       % latin1/latin9 f�r unixoide Systeme
                                                       % (latin1 ist auch unter Win verwendbar)
                                                       % ansinew f�r Windows
                                                       % applemac Macs
                                                       % cp850 OS/2
\usepackage{times}              											 % Schriften Paket
\usepackage{array,ragged2e} 													 % Wichtig f�r Abstandsformatierung

%---------------------------------------------------------------------------------------------------
\usepackage{cmbright}                                  % serifenlose Schrift als Standard
                                                       % + alle f�r TeX ben�tigten mathematischen
                                                       %   Schriften einschlie�lich der AMS-Symbole
\usepackage[scaled=.90]{helvet}                        % skalierte Helvetica als \sfdefault
\usepackage{courier}                                   % Courier als \ttdefault

%---------------------------------------------------------------------------------------------------
\usepackage[automark]{scrpage2}                        % Anpassung der Kopf- und Fu�zeilen
\usepackage{xspace}                                    % Korrekter Leerraum nach Befehlsdefinitionen
\usepackage{setspace}																	 % Dieses Package brauchen wir f�r den 				
\usepackage[pdftex]{graphicx}
\usepackage[absolute,overlay]{textpos}         
\usepackage[final]{pdfpages}																											 % anderthalbzeiligen Abstand.
\usepackage{natbib}                                    % Neuimplementierung des \cite-Kommandos
\usepackage{bibgerm}       											       % Deutsche Bezeichnungen
\usepackage[absolute]{textpos}                         % placing boxes at absolute positions
\usepackage[final]{pdfpages}                           % include pages of external PDF documents
\usepackage{tabularx}                                  % Spaltenbreite bis zur Seitenbreite dehnen
\usepackage{makeidx}
\usepackage{listings}													% Paket zur Erstellung eines Stichwortverzeichnisses
\makeindex																						 % Automatische Erstellung des Stichwortverzeichnis
\usepackage[intoc,
						english,
						prefix]{nomencl}
\makenomenclature

%---------------------------------------------------------------------------------------------------
 \usepackage{graphicx}                                 % Zur Einbindung von PDF-Bildern
 \usepackage[colorlinks,															 % Einstellen und Laden des Hyperref-Pakets
	pdftex,
	bookmarks,
	bookmarksopen=false,
	bookmarksnumbered,
	citecolor=blue,
	linkcolor=blue,
	urlcolor=blue,
	filecolor=blue,
	linktocpage,
  pdfstartview=Fit,                                  % startet mit Ganzseitenanzeige    
	pdfsubject={Multiagentensystemgest�tzte Clusteranalyse},
	pdftitle={Diplomarbeit im Fachbereich Elektrotechnik \& Informatik an der HAW-Hamburg},
	pdfauthor={Bj�rn Jensen, http://www.mirou.de}]{hyperref}
 \pdfcompresslevel=9
 
%---------------------------------------------------------------------------------------------------
% Inhaltsverzeichnis und Abschnittnummerierung
%---------------------------------------------------------------------------------------------------
\setcounter{secnumdepth}{2}   % Ich habe recht kurze Kapitel. Die sollen nicht durchnummeriert sein.
\setcounter{tocdepth}{2}

%---------------------------------------------------------------------------------------------------
% Abbildungsverzeichnis
%---------------------------------------------------------------------------------------------------
\graphicspath{{graphics/}}

%---------------------------------------------------------------------------------------------------
% Kopf- und Fu�zeilen
%---------------------------------------------------------------------------------------------------
\pagestyle{scrheadings}
\clearscrheadings
\clearscrplain
\clearscrheadfoot
\ohead{\pagemark}
\ihead{\headmark}

%---------------------------------------------------------------------------------------------------
% Neue Befehle
%---------------------------------------------------------------------------------------------------
%---------------------------------------------------------------------------------------------------
% Neue Befehle
%---------------------------------------------------------------------------------------------------

%---------------------------------------------------------------------------------------------------
% Umbenennen des Symbolverzeichnisses
%---------------------------------------------------------------------------------------------------
\renewcommand{\nomname}{Glossar}				% Das Symbolverzeichnis heisst nun "Glossar"
\renewcommand{\nomlabel}[1]{						% Die zu erkl�renden Begriffe sind nun fett hervorgehoben
	\hfil \textbf{#1} \hfil
}

%---------------------------------------------------------------------------------------------------
% Ein paar ganz n�tzliche Befehle von Lars M�hlmann
%---------------------------------------------------------------------------------------------------
%f�r Kommentare 
\newcommand{\colb}{\color{green}}
\newcommand{\colbl}{\color{black}}

%---------------------------------------------------------------------------------------------------
% Befehle zum Erstellen des Index
% \addIndexEntry{Eintrag in den Index}
% \addSubIndexEntry{Eintrag in den Index}{Eintrag des �bergeordneten Eintrags}
%---------------------------------------------------------------------------------------------------
\newcommand{\addIndexEntry}[1]{#1\index{#1}}
\newcommand{\addSubIndexEntry}[2]{#1\index{#2!#1}}

%---------------------------------------------------------------------------------------------------
% LaTeX in eigenem Font
%---------------------------------------------------------------------------------------------------
\newcommand{\myLatex}{
	{\rmfamily\LaTeX\xspace}
}

%---------------------------------------------------------------------------------------------------
% Befehl zum Erstellen und Hervorheben eines Zitats
% Parameter:
% 1. Zitat
% 2. Author
% 3. Quelle
%---------------------------------------------------------------------------------------------------
\newcommand{\myCitation}[3]{
	\begin{flushright}
	\begin{minipage}{.4\linewidth}
		\footnotesize\rmfamily\itshape 
		#1 \\
		\RaggedLeft #2 \\
		#3
	\end{minipage}
	\end{flushright}
	\nobreakspace
}

%---------------------------------------------------------------------------------------------------
% Erstellung von Deckblatt (Seite 1) und Titelblatt (Seite 2)
%---------------------------------------------------------------------------------------------------
\newcommand{\createCoverAndTitlePage}[7]{
	\createCover{#1}{#2}{#3}{#4}
	\createTitlePage{#1}{#2}{#3}{#4}{#5}{#6}{#7}
}

%---------------------------------------------------------------------------------------------------
% Erstellung von Deckblatt (Seite 1) 
% Anwendung:
% \createCover{Art der Arbeit}{Typ der Arbeit}{Autor}{Titel}
%---------------------------------------------------------------------------------------------------
\newcommand{\createCover}[4]{
	\thispagestyle{empty}
	\begin{titlepage}

	\setlength{\TPHorizModule}{1mm}
	\setlength{\TPVertModule}{1mm}
	\textblockorigin{0mm}{0mm} % start everything near the top-left corner

	% Art der Arbeit
	\begin{textblock}{111}(83,115)
		\begin{minipage}[c][1,78cm][c]{11,09cm}		
  		\fontsize{22pt}{20pt}
  		\selectfont
  		\begin{center}
  		#1#2
  		\end{center}
		\end{minipage}
	\end{textblock}

	% Name & Titel
	\begin{textblock}{111}(83,131)
		\begin{minipage}[c][4,81cm][t]{11,09cm}	
		\linespread{1.2}	
    		\fontsize{16pt}{14pt}    
    		\selectfont
    		\begin{center}
    			#3 \\ \medskip    
    			#4
    		\end{center}
    		\end{minipage}
	\end{textblock}
	\begin{textblock}{111}(35,260)
		\begin{minipage}[c][1,5cm][t]{7,0cm}		
  		\fontsize{10pt}{10pt}
  		\selectfont
		\textit{
  		Fakult\"at Technik und Informatik \\
  		Department Informations- und \\
		Elektrotechnik}
		\end{minipage}
	\end{textblock}


	\begin{textblock}{111}(125,260)
		\begin{minipage}[c][1,5cm][t]{7,0cm}		
  		\fontsize{10pt}{10pt}
  		\selectfont
		\textit{
  		Faculty of Engineering and Computer Science\\
  		Department of Information and \\
		Electrical Engineering}
		\end{minipage}
	\end{textblock}

	\end{titlepage}
%---------------------------------------------------------------------------------------------------
% Wichtig! Entsprechendes Auskommentieren!
%---------------------------------------------------------------------------------------------------
 \includepdf{pdf/DeckblattFarbe} 		% zum Ausdruck auf blanko Papier
  % **************************************************************************************
  % Originaldeckblatt mit WORD Vorlage drucken, da nur dort offizieller Schrifttyp vorhanden
  % **************************************************************************************
}

%---------------------------------------------------------------------------------------------------
% Erstellung von Titelblatt (Seite 2) 
% Anwendung:
% \createTitlePage{Art der Arbeit}{Typ der Arbeit}{Author}{Titel}{Studiengang}{Erstpr�fer}{Zweitpr�fer}
%---------------------------------------------------------------------------------------------------
\newcommand{\createTitlePage}[8]{
	\thispagestyle{empty}

	\setlength{\TPHorizModule}{1mm}
	\setlength{\TPVertModule}{\TPHorizModule}
	\textblockorigin{0mm}{0mm} % start everything near the top-left corner

	% Name & Titel
	\begin{textblock}{130}(40,63)
		\begin{minipage}[c][5,9cm][t]{13cm}
			\begin{center}
			\linespread{1.2}
			\fontsize{18pt}{18pt}
  		\selectfont
  		#3 \\ \medskip
  		\fontsize{16pt}{16pt}
  		#4
  		\end{center}
		\end{minipage}  	
	\end{textblock}

	% Infos zur Arbeit und zum Deapratment
	\begin{textblock}{126}(32,214)
  	\begin{minipage}[t][6,72cm][l]{12,57cm}
    	\fontsize{12pt}{12pt}
    	\selectfont
    	#1#2based on the study regulations\\
	for the #1 of Engineering degree programme \\
    	#5 \\
			at the Department of Information and Electrical Engineering\\
			of the Faculty of Engineering and Computer Science\\
			of the Hamburg University of Aplied Sciences\\
			\\\
			Supervising examiner : #6 \\
			Second Examiner : #7 \\
			\\\
			Day of delivery \today
  	\end{minipage}
	\end{textblock}
	\	% WICHTIG! Damit wird nach dem Titelblatt eine neue Seite angefangen! Sonst werden Titelblatt &
  	% Danksagung auf eine Seite gedruckt!
}
%---------------------------------------------------------------------------------------------------
% Erstellung von Titelblatt (Seite 2) 
% Anwendung:
% \createAbstract{Art der Arbeit}{Typ der Arbeit}{Author}{Titel}{Titel Englisch}{Stichworte}{Keywords}{Zusammenfassung}{Abstract}
%---------------------------------------------------------------------------------------------------
\newcommand{\createAbstract}[9]{
	\newpage
	\thispagestyle{empty}

	\subsection*{#3}

	\abstractentry{Title of the  #1#2}{#5}
	\abstractentry{Keywords}{#7}
	\abstractentry{Abstract}{#9}

	\selectlanguage{ngerman}
	\subsection*{#3}
	\abstractentry{Titel der Arbeit}{#4}
	\abstractentry{Stichworte}{#6}
	\abstractentry{Kurzzusammenfassung}{#8}
	\selectlanguage{english}
	\
}
%---------------------------------------------------------------------------------------------------
%Declaration
%---------------------------------------------------------------------------------------------------
\newcommand{\asurency}{
	\chapter*{Declaration}
	\vfill
	I declare within the meaning of section 25(4) of the Ex-amination and Study Regulations of the International De-gree Course Information Engineering that: this Bachelor report has been completed by myself inde-pendently without outside help and only the defined sources and study aids were used. Sections that reflect the thoughts or works of others are made known through the definition of sources.   
	\vfill
	\begin{tabularx}{\linewidth}{X l X}
	Hamburg, \today	& \qquad \qquad \qquad	& \\
	\cline{1-1}
	\cline{3-3}
	City, Date	& \qquad \qquad \qquad	& sign \\
	\end{tabularx}
	\vfill
	\vfill
	\vfill
}

%---------------------------------------------------------------------------------------------------
% F�gt ein Wort dem Index zu
%---------------------------------------------------------------------------------------------------
\newcommand{\toIndex}[1]{#1\index{#1}}

%---------------------------------------------------------------------------------------------------
% Dient zum Eintragen folgender Dinge in die Zusammenfassung (Abstract):
%	- Thema
% - Stichworte
% - Kurzfassung
% Benutzung wie folgt:
% \abstractentry{Titel}{Text}
%---------------------------------------------------------------------------------------------------
\newcommand{\abstractentry}[2]{
	\textbf{\large#1}\\ 
	\nobreakspace 
	\begin{tabular}{lp{142mm}}
		\hspace*{7mm} & #2 \\
	\end{tabular}
	\vfill
}

%---------------------------------------------------------------------------------------------------
% Erstellt eine Defintion
% Anwendung: \definition{Die Definition}
%---------------------------------------------------------------------------------------------------
\newcommand{\definition}[1]{
\begin{tabular}[ht]{lp{135mm}}
	\textbf{Def.:} & #1 \\
\end{tabular} 
}

%---------------------------------------------------------------------------------------------------
% Erstellt eine Widmung
% Anwendung: \dedication{Wem ist das Schriftst�ck gewidmet}
%---------------------------------------------------------------------------------------------------
\newcommand{\createDedication}[1]{
	\newpage
	\thispagestyle{empty}
	\begin{tabular}{lp{60mm}}
		\hspace*{100mm} & \itshape\rmfamily#1 \\
	\end{tabular}
	\vfill
}

%---------------------------------------------------------------------------------------------------
% H�ufig verwendete Namen mit Literaturverweis und Indexeintrag
%--------------------------------------------------------------------------------------------------
\newcommand{\butrynowski}{Christian Butrynowski\index{Butrynowski, Christian} \citep{Butrynowski:2005}\xspace}
\newcommand{\luepke}{Andr� L�pke\index{L�pke, Andr�}\citep{Luepke:2004}\xspace}
\newcommand{\bresch}{Marco Bresch\index{Bresch, Marco} \citep{Bresch:2004}\xspace}


%---------------------------------------------------------------------------------------------------
% Die folgenden Befehle wurden aus der Vorlage von Michael Knop �bernommen
%--------------------------------------------------------------------------------------------------
%---------------------------------------------------------------------------------------------------
% Ident
%---------------------------------------------------------------------------------------------------
\newcommand{\ident}[1]{                             % ein Parameter
	\small\ttfamily#1\sffamily\normalsize
}

%---------------------------------------------------------------------------------------------------
% K�rzel
%---------------------------------------------------------------------------------------------------
% Hier sind Makros definiert, die die Eingabe erleichtern sollen. F�r korrekte Abst�nde zwischen
% "z.B." sorgt also ein "z.\,B." (LaTeX-Befehl f�r kleineren Abstand)
% Schneller schreibt sich das durch das Makro "\zB":

% \newcommand{\zB}{z.\,B.\ }

% Hier ist der Rest aber mit dem Paket xspace verwirklicht. Damit kann
% man bei Bedarf den Abstand mit "\hspace" exakt eingeben. Dann zeigt
% LaTeX keine Toleranz bei den Abk�rzungen und macht eben exakt das
% untenstehende. 

%\renewcommand{\entryname}{K\"urzel}
%\renewcommand{\descriptionname}{Beschreibung}

\newcommand{\vgl}{vgl.\@\xspace} 
\newcommand{\abb}{Abb.\@\xspace} 
\newcommand{\zB}{z.\nolinebreak[4]\hspace{0.125em}\nolinebreak[4]B.\@\xspace}
\newcommand{\bzw}{bzw.\@\xspace}
\newcommand{\dahe}{d.\nolinebreak[4]\hspace{0.125em}h.\nolinebreak[4]\@\xspace}
\newcommand{\etc}{etc.\@\xspace}
\newcommand{\bzgl}{bzgl.\@\xspace}
\newcommand{\so}{s.\nolinebreak[4]\hspace{0.125em}\nolinebreak[4]o.\@\xspace}
\newcommand{\iA}{i.\nolinebreak[4]\hspace{0.125em}\nolinebreak[4]A.\@\xspace}
\newcommand{\sa}{s.\nolinebreak[4]\hspace{0.125em}\nolinebreak[4]a.\@\xspace}
\newcommand{\su}{s.\nolinebreak[4]\hspace{0.125em}\nolinebreak[4]u.\@\xspace}
\newcommand{\ua}{u.\nolinebreak[4]\hspace{0.125em}\nolinebreak[4]a.\@\xspace}
\newcommand{\og}{o.\nolinebreak[4]\hspace{0.125em}\nolinebreak[4]g.\@\xspace}

\newcommand{\HAW}{Hochschule f�r Angewandte Wissenschaften Hamburg\xspace}
\newcommand{\GNU}{GNU\xspace}
\newcommand{\GPL}{\GNU Public License\xspace}

\newcommand{\ACM}{ACM\xspace}
\newcommand{\PDA}{PDA\xspace}


%---------------------------------------------------------------------------------------------------
% Trennung
%---------------------------------------------------------------------------------------------------
\input{einstellungen/trennungen.sty}

%---------------------------------------------------------------------------------------------------
% Anpassung der Parameter, die TeX bei der Berechnung der Zeilenumbr�che verwendet:
%---------------------------------------------------------------------------------------------------
\tolerance 1414
\hbadness 1414
\emergencystretch 1.5em
\hfuzz 0.3pt
\widowpenalty=10000
\vfuzz \hfuzz
\raggedbottom