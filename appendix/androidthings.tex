\section{AndroidThings}
    \subsection{Installing Android Things And Start New Project}
        To get started with the project it is necessary to flash android things in the
        microcontroller. The steps to install and run android things are as follows:
        \begin{enumerate}
            \item 
                \textbf{Download and install the latest Android Things system image}
                    follow the instructions in this \href{https://developer.android.com/things/console/create.html}
                    {link}
            \item 
                \textbf{Build factory images that contain OEM applications along with the system image}
                        Configure the android things product from this \href{https://developer.android.com/things/console/configure.html}
                        {tutorial}
            \item 
            \textbf{Flash the image to raspberry pi}
                follow this you tube
                \href{https://youtu.be/9_ePSCjrQsQ?t=21}{tutorial} after the build is downloaded by following the previous steps.
                Note: This link starts from 0:21 and the process before this time is obsolete
                please do not follow the image downloading part form the tutorial.
                In case your SD card gets corrupted follow this \href{https://youtu.be/cguJpeDRfbc?list=PLr1nMHB-ifhF_3Q92jFqGUgqfABwFIKgZ}
                {tutorial}.               
            \item 
                \textbf{Download android studio}
                    after downloading android studio either you can just make a android project 
                    with an empty activity or use the template from this \href{https://github.com/androidthings/new-project-template}
                    {link}.
                    If you make choose to make your project with android studio instead of the
                    template, make sure in the app/build.gradle file you have the android things
                    dependency listed like the example \ref{code:gradleExample}.
            \item 
                \textbf{Connect to the internet to program the pi}
                    internet is required to program the raspberry pi with android things.
                    follow this \href{https://developer.android.com/things/hardware/raspberrypi.html}
                    {link}.                                  
        \end{enumerate}
    \newpage    
    \begin{lstlisting}[
        caption={Android studio gradle file example},
        label={code:gradleExample},
        language=java
        ]
        apply plugin: 'com.android.application'
        apply plugin: 'com.neenbedankt.android-apt'
        android {
            compileSdkVersion 26
            buildToolsVersion "26.0.2"
            defaultConfig {
                applicationId "navigation.egd.haw.egd_navigation_cj2"
                minSdkVersion 24
                targetSdkVersion 26
                versionCode 1
                versionName "1.0"
                testInstrumentationRunner "android.support.test.
                runner.AndroidJUnitRunner"
            }
            buildTypes {
                release {
                    minifyEnabled false
                    proguardFiles getDefaultProguardFile('proguard-android.txt'),
                     'proguard-rules.pro'
                }
            }
        }
        
        dependencies {
            compile fileTree(dir: 'libs', include: ['*.jar'])
            androidTestCompile('com.android.support.test.espresso:
            espresso-core:2.2.2', 
            {
                exclude group: 'com.android.support', module: 
                'support-annotations'
            })
            compile 'com.android.support:appcompat-v7:26.+'
            compile 'com.android.support.constraint:constraint-layout:1.0.2'
            testCompile 'junit:junit:4.12'
        
            //Android Things
            //here should be the version of android things you want to use
            provided 'com.google.android.things:androidthings:0.5.1-devpreview'
        
            //network
            compile 'com.squareup.retrofit2:retrofit:2.1.0'
            compile 'com.squareup.retrofit2:converter-gson:2.1.0'
        
            // Dependency Injection
            compile "com.google.dagger:dagger:2.6"
            provided 'javax.annotation:jsr250-api:1.0'
            apt "com.google.dagger:dagger-compiler:2.6"
        
            //Xml Reader
            compile('org.simpleframework:simple-xml:2.7.+'){
                exclude module: 'stax'
                exclude module: 'stax-api'
                exclude module: 'xpp3'
            }
        }
         
    \end{lstlisting} 