%---------------------------------------------------------------------------------------------------
% Analysis
%---------------------------------------------------------------------------------------------------
\newpage
\chapter{Requirement Analysis}
    The first phase of our project started off as a brainstorming stage where the functionality
    of the EGD was uncertain. Many discussions including a visit to the \textbf{dialog in the dark} 
    was conducted, where some questions were addressed to the visually impaired people to get an clear insight
    of what would they expect from EGD navigation system. 
\par
    From the discussions that were carried out, the requirements for the direction server were
    categorized into non-functional and functional parts which are mentioned below.
    
    \section{Functional Requirements}
        This section gives a detailed description of the functional aspects of the my work package % put a reference to this section
        and the direction API server in a tabular format in section \ref{ssec:FuncList}.
        \subsection{Functional requirements description}
            \label{ssec:FuncList}
            \begin{table}[h!]
                \centering
                    \begin{tabular}{|p{1cm}||p{15cm}|}
                        \hline
                            \textbf{No.} & \textbf{Requirement} \\
                        \hline
                            1. & Choose an appropriate operating system for the raspberry pi3 for implementing 
                            the navigation module i.e. raspbian \cite{raspbien}, 
                            android things \cite{androidThings} etc.\\
                        \hline
                            2. & Choose a software Architecture i.e. MVC \cite{mvc}, MVP \cite{mvp}
                            etc. to support the further development process.\\ 
                        \hline
                            3. & Choose an API which calculates the routes for the navigation i.e. Google Direction API \cite{googleDirecAPI}, mapbox \cite{mapbox} \\     
                        \hline
                            4. & Make an interface which can communicate to the API and return the result with a 
                            timeout to the server of 10s max, so that user gets immediate feedback. \\    
                        \hline  
                            5. & Apply Inversion of control container and use dependency Injection 
                            \cite{Martinfowler2014} framework to decouple the  dependencies 
                            in the project.\\
                        \hline   
                            6. & Run edge test cases to the API interface so that it can handle
                            errors.\\    
                        \hline    
                    \end{tabular}
                    \caption{List of functional requirements}
                    \label{table:functionalRequirements}
            \end{table}  

    \section{Non-functional Requirements}
        The requirements specified in this section present us the non-functional aspects of the 
        direction server. A tabular description is given below in the subsection 
        \ref{ssec:nonFuncList}. The detailed description depicts the benchmarks of how the system
        should be designed to meet the needs for a better sustainable prototype.

        \subsection{Non-Functional requirements description}
            \label{ssec:nonFuncList}
            \begin{table}[h!]
                \centering
                    \begin{tabular}{|p{0.5cm}||p{3cm}|p{11cm}|}
                        \hline
                            \textbf{No.} & \textbf{Requirement} & \textbf{Description} \\
                        \hline
                            1. &  Reliability & The system should provide the
                            correct walking routes. It should always take you through safe paths for 
                            walking.\\
                        \hline
                            2. & Usability & The system should be easy to use. This means the routes can be
                            calculated with just some simple parameters like start and end destination request.\\
                            
                        \hline
                            3. & Scalability & The server should be able to deal well with increase of use by many user,
                            and requesting new additional parameters i.e. change the return data to German etc., 
                            should not break the system.\\    
                        
                        \hline
                            4. & Maintainability & This relates to how the program is written and it 
                            should be designed using the software design patterns. So that it is easier to
                            debug and extend the app later.\\
                        \hline    
                            5. & Modifiability & It could be the case that the server chosen does not 
                            fulfill the use case later and the system should be decoupled.\\
                        \hline
                            6. & Responsiveness & The server should give some kind of feedback to the user
                            never the less if request cannot be made by providing an error log in case of
                            failure\\
                        \hline
                            7. & Migration of the app & Currently the proposal of the navigation module is to be
                            developed in a raspberry pi 3  and make it easier to port to a mobile device.\\ 
                                 
                        \hline    
                    \end{tabular}    
                \caption{List of non-functional requirements}
                \label{table:nonfunctionalRequirements}
            \end{table}   
