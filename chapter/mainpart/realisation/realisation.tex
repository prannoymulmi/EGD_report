%---------------------------------------------------------------------------------------------------
% Realisation
%---------------------------------------------------------------------------------------------------
\newpage
\chapter{Implementation}
    \section{Installing Android Things And Start New Project}
        To get started with the project it is necessary to flash android things in the
        microcontroller. The steps to install and run android things are as follows:
        \begin{enumerate}
            \item 
                \textbf{Download and install the latest Android Things system image}
                    follow the instructions in this \href{https://developer.android.com/things/console/create.html}
                    {link}
            \item 
                \textbf{Build factory images that contain OEM applications along with the system image}
                        Configure the android things product from this \href{https://developer.android.com/things/console/configure.html}
                        {tutorial}
            \item 
            \textbf{Flash the image to raspberry pi}
                follow this you tube
                \href{https://youtu.be/9_ePSCjrQsQ?t=21}{tutorial} after the build is downloaded by following the previous steps.
                Note: This link starts from 0:21 and the process before this time is obsolete
                please do not follow the image downloading part form the tutorial.
                In case your SD card gets corrupted follow this \href{https://youtu.be/cguJpeDRfbc?list=PLr1nMHB-ifhF_3Q92jFqGUgqfABwFIKgZ}
                {tutorial}.               
            \item 
                \textbf{Download android studio}
                    after downloading android studio either you can just make a android project 
                    with an empty activity or use the template from this \href{https://github.com/androidthings/new-project-template}
                    {link}.
                    If you make choose to make your project with android studio instead of the
                    template, make sure in the app/build.gradle file you have the android things
                    dependency listed like the example \ref{code:gradleExample}.                        
        \end{enumerate}
    \section{Google API Service Implementation}
    \section{Dependency Injection With Dagger}













