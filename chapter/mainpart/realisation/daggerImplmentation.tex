\newpage
\section{Dependency Injection With Dagger}
    This section is focused in explaining how dependency injection is implemented
    using the framework Dagger.
    The \href{https://github.com/slidenerd/Vivz_Dagger_2_Demo}{link} has demos and snippets
    of how to add dagger as dependency to your project. There are three steps considered
    to achieve dependency Injection using dagger. The described procedure is the most
    basic steps needed. 
    \begin{enumerate}
        \item 
            \textbf{Create a class which the objects that is to be injected(Module)}
                Section \ref{code:moduleDaggerExample} shows an example of class which instantiates 
                objects which would be injected. @Module defines the class and @Provides
                defines the method which provide the dependencies to be injected. Without the
                @Provides annotation the object will not be injected. 
                It is best practice to make the modules small, and only define 
                the objects in the a here 
                which belong together i.e. NetworkCheckUtilModule would only have
                objects that NavigationCheck class would require.  
        \item 
            \textbf{Create a Component class}
                A component(@Component) is an interface for dagger where one  must define 
                what modules to choose and which class should these dependencies be injected into. 
                Section \ref{code:componentDaggerExample} shows an example of the implementation of a 
                component class.
                Through the inject method you define
                which classes receive the dependencies. A Component class can contain 
                many modules, this is done so that the modules could be reusable. 
        \item 
            \textbf{Inject the dependencies}
                The last thing to be done is to inject the dependencies. This is done by
                using the annotation @Inject and then in the constructor run the builder
                of the corresponding component see section \ref{code:injectionDaggerExample} for the
                implementation. 
    \end{enumerate}
    For more detailed information for the annotations check out
    this \href{http://www.vogella.com/tutorials/Dagger/article.html}
    {link}.