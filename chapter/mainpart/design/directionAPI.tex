\section{Choosing Direction API}  
        The most popular APIs which can calculate the direction routes are Mapbox \cite{mapbox} 
        and Google Direction API \cite{googleDirecAPI}. One can make a http request with the 
        right parameter and it returns the appropriate result like below \ref{code: googleAPi Result}.
        \begin{lstlisting}[
            caption={Direction API result example}, 
            label={code: googleAPi Result},
            ]
            "legs" : [
            {
               "distance" : {
                  "text" : "35.9 mi",
                  "value" : 57824
               },
               "duration" : {
                  "text" : "51 mins",
                  "value" : 3062
               },
               "end_address" : "Hauptbahnhof, Hamburg, Germany",
               "end_location" : {
                  "lat" : 34.1330949,
                  "lng" : -118.3524442
               },
               "start_address" : "Berliner Tor 7, Hamburg, Germany",
               "start_location" : {
                  "lat" : 33.8098177,
                  "lng" : -117.9154353
               },  
        \end{lstlisting}
    
        Google Direction API deemed to be the best choice as the API which
        would calculate the routes that would be required for navigation.
        The reasons for choosing it over Mapbox API are as follows:

        \begin{enumerate}  
            \item 
                Google has huge data streams. This means that google API can provide more
                accurate directions since it has an enormous amount of data collect. The amount
                of data google has cannot be compared with Mapbox.
            \item
                It provides a free limit of 2500 direction request per day which is more than enough for testing our prototype.
                compared to  50,000 views/month.
            \item
                Google has a very well and up to date documentation of how to use the API.  
            \item
                Google API has a very easy to use interface meaning the input 
                parameters that it expects are very simple. 
                A sample request can be seen in the code \ref{code: googleAPIRequest}
                \begin{lstlisting}[
                    caption={Sample request to Google direction API}, 
                    label={code: googleAPIRequest},
                    language=c
                    ]
                    https://maps.googleapis.com/maps/api/directions/json?
                    origin=Disneyland
                    &destination=Universal+Studios+Hollywood4
                    &key=YOUR_API_KEY
                \end{lstlisting}   
            \item 
                Google has developed a vast range of web services that can be
                combined with \href{https://developers.google.com/maps/documentation/directions/start}
                {Google direction API} such as \href{https://developers.google.com/places/} 
                {Places API}, \href{https://developers.google.com/maps/documentation/geolocation/intro}
                {Geolocation API}, \href{https://developers.google.com/maps/documentation/elevation/start}
                {Elevation API} and \href{https://developers.google.com/places/documentation/} {many others}. 
                These services can be combined later for a better results and mapbox 
                currently does not provide these services. 
        \end{enumerate}