\section{Choosing Direction API}  
        The most popular API's which can calculate the direction routes are Mapbox \cite{mapbox} 
        and Google Direction API \cite{googleDirecAPI}. One can make a http request with the 
        right parameter and it returns the appropriate result similar to Listing 
        \ref{code: googleAPi Result}.
        \begin{lstlisting}[
            caption={Direction API result example}, 
            label={code: googleAPi Result},
            ]
            "legs" : [
            {
               "distance" : {
                  "text" : "35.9 mi",
                  "value" : 57824
               },
               "duration" : {
                  "text" : "51 mins",
                  "value" : 3062
               },
               "end_address" : "Hauptbahnhof, Hamburg, Germany",
               "end_location" : {
                  "lat" : 34.1330949,
                  "lng" : -118.3524442
               },
               "start_address" : "Berliner Tor 7, Hamburg, Germany",
               "start_location" : {
                  "lat" : 33.8098177,
                  "lng" : -117.9154353
               },  
        \end{lstlisting}
    
        \subsection{Evaluation}
            Google Direction API seems to be the better choice as the 
            direction API.
            The reasons for choosing it over Mapbox API are as follows:

            \begin{enumerate}  
                \item 
                    Google has huge data streams. This means that google API can provide more
                    accurate directions since it has an enormous data. The quantity
                    of data google has cannot be compared with Mapbox.
                \item
                    It provides a free limit of 2500 direction request per day which is more than enough for testing our prototype.
                    compared to  50,000 views/month.
                \item
                    Google has a very well and up to date documentation for its API.  
                \item
                    Google API has a very easy to use interface meaning the input 
                    parameters that it expects are very simple. 
                    A sample request can be seen in the code \ref{code: googleAPIRequest}
                    \begin{lstlisting}[
                        caption={Sample request to Google direction API}, 
                        label={code: googleAPIRequest},
                        language=c
                        ]
                        https://maps.googleapis.com/maps/api/directions
                        /json?origin=Disneyland
                        &destination=Universal+Studios+Hollywood4
                        &key=YOUR_API_KEY
                    \end{lstlisting}   
                \item 
                    Google has developed a vast range of web services that can be
                    combined with
                    Google direction API such as  
                    Places API, Geolocation API,
                    Elevation API and many others \cite{GoogleWebServices}. 
                    These services can be combined later for a better results and mapbox 
                    currently does not provide these services. 
            \end{enumerate}