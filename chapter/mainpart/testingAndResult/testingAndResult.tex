\chapter{Testing and Validation}
Some scenarios were tested to verify the
correct functionality of the Google Direction
API. The tests were made in accordance
to the functional requirements mentioned
in section \ref{ssec:FuncList}. The implementation for 
catching the errors are implemented but due to the fact that,
how the error handling must be done was not discussed a null (where possible) 
is being returned with comments in the code that, it is to be decided later. 

\section{Invalid Input Handling Test}
    This test is specifically to evaluate what would happen if a wrong Input address i.e. 
    Berliner Dor 7 instead of Berliner Tor 7 is received. If the provided input does not exist
    the server still returns a success code of 200 but the attribute \textbf{status} inside the body
    would not be ok. See method getWalkingDirection in appendix 
    \ref{code:retrofitConfig}, this is where the check is done to verify
    if the result received is valid. In this 
    \href{https://developers.google.com/maps/documentation/javascript/directions#DirectionsRegionBiasing} 
    {link} are the detailed description of the status codes return by this API.

\section{Inputting an Address which exist in Another city}
    This is a test to check the behavior in the case, when the 
    end address provided is of the same name in another city or country. 
    Example of the test case is if a user only gives \textbf{Rathhaus} 
    as an input the API does not necessarily understands it as Rathaus Hamburg since
    even though the request are being made from Hamburg.
    To solve this issue the end destination address should not only
    be the Rathaus rather Rathaus Hamburg and the optional
    query region must be also sent to ensure the correct
    result. 
    
    \textbf{Note: }To pass in additional optional queries a Map<String, Sting>
    is being passed see appendix \ref{code:retrofitConfig} method getWalkingDirection
    i.e. Map<"region", "DE"> any optional queries can be passed like this.

\section{Network Timeout Test}
    This test is done to check the behavior of the system
    if the API fails to respond within the
    given timeframe. According to the requirements
    the timeout is supposed to be 10s but during the testing phase 
    the server always responded within this timeframe therefore the
    timeout was reduced to 1s for testing. After
    reducing the timeframe the server could not respond within this time 
    and a SocketTimeoutException was thrown which can be handled easily. See
    appendix
    \ref{code:retrofitConfig} how it is being handled. 

    Since the timeout works for 1s it will also work for 10s since its just a arbitrary
    time the task will wait until terminating the call since the API is not responding. 

\section{Internet Connection Availability Test}
    \label{sec:internetConnectionTest}
    To make a API request a internet connection is necessary. There is 
    already a check being made which is being done by
    NetworkCheckUtil for active internet connection see appendix
    \ref{code:networkCheckUtil} for its implementation.

    This test case could not be tested because an active internet connection
    is needed to debug the microcontroller (Android things bottle neck). Due to this reason this test could not
    be fully tested.
    